\documentclass{article}

\usepackage{fancyhdr}
\usepackage{extramarks}
\usepackage{boondox-cal}
\usepackage{amsmath}
\usepackage{amsthm}
\usepackage{amsfonts}
\usepackage{tikz}
\usepackage[plain]{algorithm}
\usepackage{algpseudocode}

\usetikzlibrary{automata,positioning}

%
% Basic Document Settings
%

\topmargin=-0.45in
\evensidemargin=0in
\oddsidemargin=0in
\textwidth=6.5in
\textheight=9.0in
\headsep=0.25in

\linespread{1.1}

\pagestyle{fancy}
\lhead{\hmwkAuthorName}
\chead{\hmwkClass\ (\hmwkClassInstructor): \hmwkTitle}
\rhead{\firstxmark}
\lfoot{\lastxmark}
\cfoot{\thepage}

\renewcommand\headrulewidth{0.4pt}
\renewcommand\footrulewidth{0.4pt}

\setlength\parindent{0pt}

%
% Create Problem Sections
%

\newcommand{\enterProblemHeader}[1]{
    \nobreak\extramarks{}{Problem \arabic{#1} continued on next page\ldots}\nobreak{}
    \nobreak\extramarks{Problem \arabic{#1} (continued)}{Problem \arabic{#1} continued on next page\ldots}\nobreak{}
}

\newcommand{\exitProblemHeader}[1]{
    \nobreak\extramarks{Problem \arabic{#1} (continued)}{Problem \arabic{#1} continued on next page\ldots}\nobreak{}
    \stepcounter{#1}
    \nobreak\extramarks{Problem \arabic{#1}}{}\nobreak{}
}

\setcounter{secnumdepth}{0}
\newcounter{partCounter}
\newcounter{homeworkProblemCounter}
\setcounter{homeworkProblemCounter}{1}
\nobreak\extramarks{Problem \arabic{homeworkProblemCounter}}{}\nobreak{}

%
% Homework Problem Environment
%
% This environment takes an optional argument. When given, it will adjust the
% problem counter. This is useful for when the problems given for your
% assignment aren't sequential. See the last 3 problems of this template for an
% example.
%
\newenvironment{homeworkProblem}[1][-1]{
    \ifnum#1>0
        \setcounter{homeworkProblemCounter}{#1}
    \fi
    \section{Problem \arabic{homeworkProblemCounter}}
    \setcounter{partCounter}{1}
    \enterProblemHeader{homeworkProblemCounter}
}{
    \exitProblemHeader{homeworkProblemCounter}
}

%
% Homework Details
%   - Title
%   - Due date
%   - Class
%   - Section/Time
%   - Instructor
%   - Author
%

\newcommand{\hmwkTitle}{Homework\ \#3}
\newcommand{\hmwkDueDate}{February 26, 2024}
\newcommand{\hmwkClass}{Linear Algebra}
\newcommand{\hmwkClassInstructor}{Gilbert Strang}
\newcommand{\hmwkAuthorName}{\textbf{0130}}

%
% Title Page
%

\title{
    \vspace{2in}
    \textmd{\textbf{\hmwkClass:\ \hmwkTitle}}\\
    \vspace{0.1in}\large{\textit{\hmwkClassInstructor}}
    \vspace{3in}
}

\author{\hmwkAuthorName}
\date{}

\renewcommand{\part}[1]{\textbf{\large Part \Alph{partCounter}}\stepcounter{partCounter}\\}

%
% Various Helper Commands
%

% Useful for algorithms
\newcommand{\alg}[1]{\textsc{\bfseries \footnotesize #1}}

% For derivatives
\newcommand{\deriv}[1]{\frac{\mathrm{d}}{\mathrm{d}x} (#1)}

% For partial derivatives
\newcommand{\pderiv}[2]{\frac{\partial}{\partial #1} (#2)}

% Integral dx
\newcommand{\dx}{\mathrm{d}x}

% Alias for the Solution section header
\newcommand{\solution}{\textbf{\large Solution}}

% Probability commands: Expectation, Variance, Covariance, Bias
\newcommand{\E}{\mathrm{E}}
\newcommand{\Var}{\mathrm{Var}}
\newcommand{\Cov}{\mathrm{Cov}}
\newcommand{\Bias}{\mathrm{Bias}}

\begin{document}

\maketitle

\pagebreak

\begin{homeworkProblem}
    \begin{enumerate}
        \item Which rules are broken if we keep only the positive numbers
            \( x > 0 \) in \( \mathbf{R}^1 \)? Every \( c \) must be allowed.
            This half-line is not a subspace.
        \item The positive numbers with \( \mathbf{x} + \mathbf{y} \) and
            \( c \mathbf{x} \) redefined to equal the usual \( xy \) and \( x^c \)
            do satisfy the eight rules. Test rule 7 when \( c = 3, x = 2, y = 1 \).
            (Then \( \mathbf{x} + \mathbf{y} = 2\)) and \( c\mathbf{x} = 8 . \)
            Which number acts as the ``zero vector'' in this space?
    \end{enumerate}

    \solution

    \begin{enumerate}
        \item Rule 4.
        \item 1 acts as the ``zero vector'' in this space.
    \end{enumerate}
\end{homeworkProblem}

\begin{homeworkProblem}
    \( M \) is the space of 2 by 2 matrices.
    \begin{enumerate}
        \item Describe a subspace of \( M \) that contains \( A = \begin{bmatrix}
            1 & 0   \\
            0 & 0
        \end{bmatrix}\)
        but not \( B = \begin{bmatrix}
            0 & 0   \\
            0 & -1
        \end{bmatrix} \).
        \item If a subspace of \( M \) does contain \( A \) and \( B \), must it contain
            the identity matrix?
        \item Describe a subspace of \( M \) that contains no nonzero diagonal matrices.
    \end{enumerate}

    \solution

    \begin{enumerate}
        \item \( \left\{ Q\ |\ Q = c\begin{bmatrix} 1 & 0   \\ 0 & 0\end{bmatrix}, c \in \mathbb{R} \right\} \)
        \item Yes, because identity matrix is a linear combination of \( A \) and \( B \).
        \item \( \left\{ Q\ |\ Q = \begin{bmatrix}
            0 & a   \\
            b & 0
        \end{bmatrix}, a, b \in \mathbb{R}
        \right\}\)
    \end{enumerate}
\end{homeworkProblem}

\begin{homeworkProblem}
    The columns of \( AB \) are combinations of the columns of \( A \). This means: The column space
    of \( AB \) is contained in (possibly equal to) the column space \( A \). Give an example
    where the columns spaces of \( A \) and \( AB \) are not equal.
    \\

    \solution
    \\

    We can consider the case where the matrix \( B \) is not full of rank and give
    the following example:

    \[
        A = 
        \begin{bmatrix}
            1 & 0   \\
            0 & 1
        \end{bmatrix}
        ,
        B = 
        \begin{bmatrix}
            1 & 0   \\
            0 & 0
        \end{bmatrix}
        ,
        AB = 
        \begin{bmatrix}
            1 & 0   \\
            0 & 0
        \end{bmatrix}
    \]

\end{homeworkProblem}

\pagebreak

\begin{homeworkProblem}
    (nullspace of \( A \)) Create a 2 by 4 matrix \( R \) whose
    special solutions to \( R\mathbf{x} = 0 \) are
    \( \mathbf{s}_1 \) and \( \mathbf{s}_2 \):

    \[
        \mathbf{s}_1 =
        \begin{bmatrix}
            -3&          \\
            \mathbf{1}&  \\
            0&           \\
            \mathbf{0}&
        \end{bmatrix}
        \ \ \ 
        \text{and}
        \ \ \ 
        \mathbf{s}_2 =
        \begin{bmatrix}
            -2&          \\
            \mathbf{0}&  \\
            -6&           \\
            \mathbf{1}&
        \end{bmatrix}
    \]
pivots columns 1 and 3 free variables \( x_2\) and \(x_4 \),
\( x_2 \) and \( x_4 \) are 1,0 and 0,1 in the ``special solutions''.
\\

Describe all 2 by 4 matrices with this nullspace \( \mathbf{N}(A) \) spanned by \( \mathbf{s}_1 \) and \( \mathbf{s}_2 \).
\\

\solution

\[      
    \mathbf{N}(A) = \left\{ c_1\mathbf{s}_1 + c_2\mathbf{s}_2\ |\ c_1, c_2 \in \mathbb{R}
    \right\} = \left\{
        c_1
        \begin{bmatrix}
            -3 &    \\
            \mathbf{1} & \\
            0 &   \\
            \mathbf{1}
        \end{bmatrix}
        +
        c_2
        \begin{bmatrix}
            -2 &    \\
            \mathbf{0} &  \\
            -6 &    \\
            \mathbf{1}
        \end{bmatrix}
        \ |\ 
        c_1, c_2 \in \mathbb{R}
    \right\}
\]

\end{homeworkProblem}

\begin{homeworkProblem}
    Suppose an \( m \) by \( n \) matrix has \( r \) pivots. The number
    of special solutions (basis for the nullspace) is \underline{\ \ \ \ \ \ }
    by the Counting Theorem. The nullspace contains only \( \mathbf{x = 0} \) when
    \( r = \underline{\ \ \ \ \ \ } \). The column space is all of \( \mathbf{R}^m \)
    when the rank is \( r = \underline{\ \ \ \ \ \ } \).
    \\

    \solution

    \begin{enumerate}
        \item \( n - r \)
        \item \( n \)
        \item \( m \)
    \end{enumerate}
\end{homeworkProblem}

\begin{homeworkProblem}
    Construct a matrix whose column space contains (1,1,5) and (0,3,1)
    and whose nullspace contains (1,1,2).
    \\

    \solution

    \[
        \begin{bmatrix}
            1 & 0 & -\frac{1}{2} \\
            1 & 3 & -2 \\
            5 & 1 & -3
        \end{bmatrix}
    \]
\end{homeworkProblem}

\begin{homeworkProblem}
    If \( AB = 0 \) then the column space of \( B \)
    is contained in the \underline{\ \ \ \ \ \ } of \( A \). Why?
    \\

    \solution

    The column space of \( B \) is contained in the nullspace of \( A \).
    Because if we rewrite matrix \( B \) as the following form:

    \[
        B = [B_1, B_2, \ldots , B_q]
    \]

    where the column number of \( B_i (i \in \left\{ 1, \ldots, q \right\}) \)
    is 1,then \( AB \) can be represented as \( [AB_1, AB_2, \ldots, AB_q] \),
    and we know \( B_1, \ldots, B_q \) are in the nullspace of A since \( AB = 0 \).
    Then we know the space spanned by \( B_1, \ldots, B_q \) (which is the column space
    of \( B \)) is contained in the nullspace of \( A \).

\end{homeworkProblem}

\begin{homeworkProblem}
    How is the nullspace \( \mathbf{N}(C) \) related to the spaces \( \mathbf{N}(A) \)
    and \( \mathbf{N}(B) \), if \( C = \begin{bmatrix}
        A   \\
        B
    \end{bmatrix}\)?
    \\

    \solution
    \\

    \( \mathbf{N}(C) = \mathbf{N}(A)\ \cap\ \mathbf{N}(B) \).
\end{homeworkProblem}

\begin{homeworkProblem}
    \( A = C\begin{bmatrix}
        I\ \ F
    \end{bmatrix} = C \begin{bmatrix}
        1 & 0 & 1 & 2   \\
        0 & 1 & 3 & 1
    \end{bmatrix}\).
    \( C \) has 2 independent columns.
    \\

    Find the 2 special solutions to \( A\mathbf{x} = \mathbf{0} \)
    of the form \( (x_1, x_2, 1, 0) \) and \( (x_1, x_2, 0, 1) \).
    \\

    Note: If \( A\mathbf{x} = \mathbf{b} \) has a solution \( x = x_p \) then
    all its solutions have the form \( x = x_p + x_n \). Here 
    \( x_p \) is the only solution in the row space of \( A \) and \( x_n \)
    is in the nullspace of \( A \) (so \(Ax_n = 0 \)).
    \\

    \solution
    \\

    Let \( R = \begin{bmatrix}
        1 & 0 & 1 & 2   \\
        0 & 1 & 3 & 1
    \end{bmatrix}\), then \( A\mathbf{x} = 0 \) has the same solutions of \( R\mathbf{x} = 0 \).
    We can find two special solutions by specifying the free variables to (1, 0) and (0, 1):

    \[
        \mathbf{v}_1 = \begin{pmatrix}
            -2, -3, 1, 0
        \end{pmatrix},
        \mathbf{v}_2 = \begin{pmatrix}
            -2, -1, 0, 1
        \end{pmatrix}
    \]
\end{homeworkProblem}

\begin{homeworkProblem}
    Write the complete solution as \( \mathbf{x}_p \) plus
    any multiple of \( \mathbf{s} \) in the nullspace:

    \[
        \begin{array}{ll}
            x + 3y = 7  \\
            2x + 6y = 14
        \end{array}
        \ \ \ \ \ \ 
        \begin{array}{ll}
            x + 3y + 3z = 1 \\
            2x + 6y + 9z = 5    \\
            -x - 3y + 3z = 5
        \end{array}
    \]

    \solution

    \begin{enumerate}
        \item \( \mathbf{x}_p = \begin{bmatrix}
            1   \\
            2
        \end{bmatrix} + 
        c \begin{bmatrix}
            1&   \\
            -\frac{1}{3}&
        \end{bmatrix},\ c \in \mathbb{R} \)
        \item \( \mathbf{x}_p = \begin{bmatrix}
            1&  \\
            -1& \\
            1
        \end{bmatrix} + c
        \begin{bmatrix}
            3&   \\
            -1&  \\
            0&
        \end{bmatrix},\ c \in \mathbb{R}\)
    \end{enumerate}
\end{homeworkProblem}

\begin{homeworkProblem}
    Under what conditions on \( b_1, b_2, b_3 \) are these systems solvable?
    Include \( \mathbf{b} \) as a fourth column in elimination. Find all solutions
    when that solvability condition holds:

    \[
        \begin{array}{ll}
            x + 2y - 2z = b_1   \\
            2x + 5y - 4z = b_2  \\
            4x + 9y - 8z = b_3
        \end{array}
        \ \ \ \ \ \
        \begin{array}{ll}
            2x + 2z = b_1   \\
            4x + 4y = b_2   \\
            8x + 8y = b_3
        \end{array}
    \]

    \solution

    \begin{enumerate}
        \item \( b_3 - b_2 - 2b_1 = 0\).
        \item \( b_3 - 2b_2 = 0 \).
    \end{enumerate}
\end{homeworkProblem}

\begin{homeworkProblem}
    Construct a 2 by 3 system \( A\mathbf{x} = \mathbf{b} \) with
    particular solution \( \mathbf{x}_p = \begin{pmatrix}
        2, 4, 0
    \end{pmatrix}\)
    and homogeneous solution \( \mathbf{x}_n \) = any multiple
    of (1,1,1).
    \\

    \solution

    The augmented matrix of this system is given below:

    \[
        \begin{bmatrix}
            1 & 2 & -3 & 10 \\
            4 & 3 & -7 & 20
        \end{bmatrix}
    \]
\end{homeworkProblem}

\begin{homeworkProblem}
    Give examples of matrices \( A \) for which the number of
    solutions to \( A\mathbf{x} = \mathbf{b} \) is
    \begin{enumerate}
        \item 0 or 1, depending on \( \mathbf{b} \)
        \item \( \infty \), regardless of \( \mathbf{b} \)
        \item 0 or \( \infty \), depending on \( \mathbf{b} \)
        \item 1, regardless of \( \mathbf{b} \)
    \end{enumerate}

    \solution
    \\

    The number of solutions is determined by the rank of \( A \), suppose \( A \) is a n by m matrix.

    \begin{enumerate}
        \item rank\( (A) \) = n < m: \( A = \begin{bmatrix}
            1 & 0   \\
            0 & 1   \\
            0 & 0   \\
            0 & 0
        \end{bmatrix} \)
        \item rank\( (A) \) = m < n: \(  A = \begin{bmatrix}
            1 & 0 & 0 & 0   \\
            0 & 0 & 0 & 1
        \end{bmatrix} \)
        \item rank \( (A) \) < m; rank \( (A) \) < n: \( A = \begin{bmatrix}
            1 & 0 & 0 & 0   \\
            0 & 0 & 0 & 0
        \end{bmatrix} \)
        \item rank \( (A) \) = n = m: \( A = \begin{bmatrix}
            1 & 0 & 0   \\
            0 & 1 & 0   \\
            0 & 0 & 1
        \end{bmatrix}\)
    \end{enumerate}
\end{homeworkProblem}

\begin{homeworkProblem}
    Find a basis (independent vectors that span the subspace) for each of these subspaces
    of \( \mathbf{R}^4 \):
    \begin{enumerate}
        \item All vectors whose components are equal.
        \item All vectors whose components add to zero.
        \item All vectors that are perpendicular to (1, 1, 0, 0) and (1, 0, 1, 1).
        \item The column space and the nullspace of \( I \) (4 by 4).
    \end{enumerate}

    Find a basis (and the dimension) for each of these subspaces of 3 by 3 matrices:

    \begin{enumerate}
        \item All symmetric matrices (\( A^T = A \)).
        \item All skew-symmetric matrices (\( A^T = -A \)).
    \end{enumerate}

    \solution

    \begin{enumerate}
        \item \( \begin{bmatrix}
            1   \\
            1   \\
            1   \\
            1
        \end{bmatrix}\)
        \item To get the dimension of this subspace,
        one way is to think of this subspace as a nullspace of
        \( \begin{bmatrix}
            1 & 1 & 1 & 1
        \end{bmatrix} \), and then we know the dimension of
        this subspace is 3, assign 1 to free variables one by one,
        we have:
        \[     
        \begin{bmatrix}
            -1&   \\
            1&   \\
            0&  \\
            0&
        \end{bmatrix},
        \begin{bmatrix}
            -1&  \\
            0&  \\
            1&  \\
            0&  
        \end{bmatrix},
        \begin{bmatrix}
            -1&  \\
            0&  \\
            0& \\
            1&
        \end{bmatrix}
        \]
        \item Let \( A = \begin{bmatrix}
            1 & 1 & 0 & 0   \\
            1 & 0 & 1 & 1
        \end{bmatrix}\), the question is, find a basis of the nullspace of \( A \)?
        Using elimination, a basis is given below:
        \[
            \begin{bmatrix}
                -1&  \\
                1&   \\
                1&   \\
                0&
            \end{bmatrix},
            \begin{bmatrix}
                -1& \\
                1&  \\
                0&  \\
                1&
            \end{bmatrix}
        \]
        \item \begin{enumerate}
            \item A basis of the column space of \( I \):
                \[
                    \begin{bmatrix}
                        1&  \\
                        0&  \\
                        0&  \\
                        0
                    \end{bmatrix},
                    \begin{bmatrix}
                        0&  \\
                        1&  \\
                        0&  \\
                        0
                    \end{bmatrix},
                    \begin{bmatrix}
                        0&  \\
                        0&  \\
                        1&  \\
                        0
                    \end{bmatrix},
                    \begin{bmatrix}
                        0&  \\
                        0&  \\
                        0&  \\
                        1
                    \end{bmatrix}
                \]
            \item The nullspace of \( I \) doesn't have any bases.
        \end{enumerate}
        \item Let \( S \) represents the subspace of all symmetric matrices.
        A basis of this subspace is given below:
            \[
            \begin{bmatrix}
                1 & 0 & 0   \\
                0 & 0 & 0   \\
                0 & 0 & 0
            \end{bmatrix},
            \begin{bmatrix}
                0 & 0 & 0   \\
                0 & 1 & 0   \\
                0 & 0 & 0
            \end{bmatrix},
            \begin{bmatrix}
                0 & 0 & 0   \\
                0 & 0 & 0   \\
                0 & 0 & 1
            \end{bmatrix},
            \begin{bmatrix}
                0 & 1 & 0   \\
                1 & 0 & 0   \\
                0 & 0 & 0
            \end{bmatrix},
            \begin{bmatrix}
                0 & 0 & 1   \\
                0 & 0 & 0   \\
                1 & 0 & 0
            \end{bmatrix},
            \begin{bmatrix}
                0 & 0 & 0   \\
                0 & 0 & 1   \\
                0 & 1 & 0
            \end{bmatrix},
            \]
        And we know dim\(( S )\) = 6.
    \item Let \( S \) represents the subspace of all skew-symmetric
    matrices, a basis of this subspace is given below: \[
            \begin{bmatrix}
                0 & -1 & 0  \\
                1 & 0 & 0   \\
                0 & 0 & 0
            \end{bmatrix},
            \begin{bmatrix}
                0 & 0 & -1  \\
                0 & 0 & 0   \\
                1 & 0 & 0
            \end{bmatrix},
            \begin{bmatrix}
                0 & 0 & 0   \\
                0 & 0 & -1  \\
                0 & 1 & 0
            \end{bmatrix}
        \]
    And we know dim\( (S) \) = 3.
    \end{enumerate}
\end{homeworkProblem}



\end{document}
