\documentclass{article}

\usepackage{fancyhdr}
\usepackage{extramarks}
\usepackage{boondox-cal}
\usepackage{amsmath}
\usepackage{amsthm}
\usepackage{amsfonts}
\usepackage{tikz}
\usepackage[plain]{algorithm}
\usepackage{listings}
\usepackage{algpseudocode}

\usetikzlibrary{automata,positioning, arrows.meta}

%
% Basic Document Settings
%

\topmargin=-0.45in
\evensidemargin=0in
\oddsidemargin=0in
\textwidth=6.5in
\textheight=9.0in
\headsep=0.25in

\linespread{1.1}

\pagestyle{fancy}
\lhead{\hmwkAuthorName}
\chead{\hmwkClass\ (\hmwkClassInstructor): \hmwkTitle}
\rhead{\firstxmark}
\lfoot{\lastxmark}
\cfoot{\thepage}

\renewcommand\headrulewidth{0.4pt}
\renewcommand\footrulewidth{0.4pt}

\setlength\parindent{0pt}

%
% Create Problem Sections
%

\newcommand{\enterProblemHeader}[1]{
    \nobreak\extramarks{}{Problem \arabic{#1} continued on next page\ldots}\nobreak{}
    \nobreak\extramarks{Problem \arabic{#1} (continued)}{Problem \arabic{#1} continued on next page\ldots}\nobreak{}
}

\newcommand{\exitProblemHeader}[1]{
    \nobreak\extramarks{Problem \arabic{#1} (continued)}{Problem \arabic{#1} continued on next page\ldots}\nobreak{}
    \stepcounter{#1}
    \nobreak\extramarks{Problem \arabic{#1}}{}\nobreak{}
}

\setcounter{secnumdepth}{0}
\newcounter{partCounter}
\newcounter{homeworkProblemCounter}
\setcounter{homeworkProblemCounter}{1}
\nobreak\extramarks{Problem \arabic{homeworkProblemCounter}}{}\nobreak{}

%
% Homework Problem Environment
%
% This environment takes an optional argument. When given, it will adjust the
% problem counter. This is useful for when the problems given for your
% assignment aren't sequential. See the last 3 problems of this template for an
% example.
%
\newenvironment{homeworkProblem}[1][-1]{
    \ifnum#1>0
        \setcounter{homeworkProblemCounter}{#1}
    \fi
    \section{Problem \arabic{homeworkProblemCounter}}
    \setcounter{partCounter}{1}
    \enterProblemHeader{homeworkProblemCounter}
}{
    \exitProblemHeader{homeworkProblemCounter}
}

%
% Homework Details
%   - Title
%   - Due date
%   - Class
%   - Section/Time
%   - Instructor
%   - Author
%

\newcommand{\hmwkTitle}{Homework\ \#5}
\newcommand{\hmwkDueDate}{February 29, 2024}
\newcommand{\hmwkClass}{Linear Algebra}
\newcommand{\hmwkClassInstructor}{Gilbert Strang}
\newcommand{\hmwkAuthorName}{\textbf{0130}}

%
% Title Page
%

\title{
    \vspace{2in}
    \textmd{\textbf{\hmwkClass:\ \hmwkTitle}}\\
    \vspace{0.1in}\large{\textit{\hmwkClassInstructor}}
    \vspace{3in}
}

\author{\hmwkAuthorName}
\date{}

\renewcommand{\part}[1]{\textbf{\large Part \Alph{partCounter}}\stepcounter{partCounter}\\}

%
% Various Helper Commands
%

% Useful for algorithms
\newcommand{\alg}[1]{\textsc{\bfseries \footnotesize #1}}

% For derivatives
\newcommand{\deriv}[1]{\frac{\mathrm{d}}{\mathrm{d}x} (#1)}

% For partial derivatives
\newcommand{\pderiv}[2]{\frac{\partial}{\partial #1} (#2)}

% Integral dx
\newcommand{\dx}{\mathrm{d}x}

% Alias for the Solution section header
\newcommand{\solution}{\textbf{\large Solution}}

% Probability commands: Expectation, Variance, Covariance, Bias
\newcommand{\E}{\mathrm{E}}
\newcommand{\Var}{\mathrm{Var}}
\newcommand{\Cov}{\mathrm{Cov}}
\newcommand{\Bias}{\mathrm{Bias}}

% empty underline
\newcommand{\emptyunderline}{\underline{\ \ \ \ \ \ }}

\begin{document}

\maketitle

\pagebreak

Before start this homework, let's review the process of Gram-Schmidt method.
So the reason why we want to get orthonormal basis is that, in the chapter
of the projection of a vector onto a subspace,  we know that if we want to project a vector
to the column space of a matrix \( A \), the corresponding projection matrix is

\[
    P = A{(A^T A)}^{-1} A^T
\]

if we have an orthonormal basis for the column space, the above formula becomes:

\[
    P = QQ^T
\]

which is much easier to compute. And remember, here the columns of \( A \) must be
linearly independent, otherwise we can't do the Gram-Schmidt process.
\\

Let's think of Gram-Schmidt process in the 2 dimension case first.
We pick the first column
vector \( \mathbf{a}_1 / || \mathbf{a}_1 || \) as the first vector \( \mathbf{e}_1 \), and try to find the second vector that is orthogonal
to the first one. What we do is to project the second column vector \( \mathbf{a}_2 \) to the first one,
decompose \( \mathbf{a}_2 \) into two parts \( \mathbf{v}_2 + \mathbf{p}_2 \), where
\( \mathbf{v}_2 \bot  \mathbf{e}_1 \) and \( \mathbf{p}_2 \in\ \text{span}(\mathbf{e}_1) \), we then have

\[  
    \begin{array}{ll}
        \mathbf{e}_1^T(\mathbf{a}_2 - \mathbf{p}_2) = 0 \\

        \mathbf{p}_2 = c_2 \mathbf{e}_1, c_2 \in \mathbb{R}
    \end{array}
\]

after some calculation, we get \( c_2 = \mathbf{e}_1^T \mathbf{a}_2 / \mathbf{e}_1^T\mathbf{e}_1 \), so \( \mathbf{v}_2 \) can be computed as:

\[ \mathbf{v}_2 =  \mathbf{a}_2 - \frac{\mathbf{e}_1^T \mathbf{a}_2}{\mathbf{e}_1^T \mathbf{e_1}} \mathbf{e}_1 = \mathbf{a}_2 - \mathbf{e}_1^T \mathbf{a}_2 \mathbf{e}_1
\]

then we can get the second vector \( \mathbf{e}_2 = \mathbf{v}_2 / || \mathbf{v}_2 || \).
\\

After discussing the 2 dimension case, we can talk about the general case which is similar to the 2-dim case.
\\

So suppose we have a sequence of vectors
\( \mathbf{e}_1, \ldots, \mathbf{e}_{n-1} \), these vectors are linearly independent and they are orthonormal, now I have a vector \( \mathbf{a}_n \) which is
also linearly independent to these vectors, how can I find a vector \( \mathbf{e}_n \) that is orthonormal to \( \mathbf{e}_1, \ldots, \mathbf{e}_{n-1} \) and we also have
span{(\( \mathbf{e}_1, \ldots, \mathbf{e}_n \))} = span{(\( \mathbf{e}_1, \ldots, \mathbf{e}_{n-1}, \mathbf{a}_n \))}?
\\

A natural idea is to throw away the part of \( \mathbf{a}_n \) that is in the span of \( \mathbf{e}_1, \ldots, \mathbf{e}_{n-1} \), as we know,
\( \mathbf{a}_n \) can we written as \( \mathbf{v}_n + \mathbf{p_n} \), define the matrix \( A_{n-1} \) equals to \( \begin{bmatrix}
    \mathbf{e}_1, \mathbf{e}_2, \ldots, \mathbf{e}_{n-1}
\end{bmatrix} \)
, then we can represent \( \mathbf{v}_n \) as \( \mathbf{v}_n = \mathbf{a}_n - A_{n-1}\mathbf{x} \), since \( \mathbf{v}_n \bot\ C(A_{n-1}) \), we know
\( \mathbf{v}_n \in\ \mathbf{N}(A_{n-1}^T) \), then we have

\[
    \begin{split}
        A_{n-1}^T\mathbf{v}_n =& A_{n-1}^T(\mathbf{a}_n - \mathbf{p}_n)  \\
                        =& A_{n-1}^T(\mathbf{a}_n - A_{n-1}\mathbf{x}) \\
                        =& 0
    \end{split}
\]

Then we get \( \mathbf{x} = {(A_{n-1}^T A_{n-1})}^{-1} A_{n-1}^T \mathbf{a}_n\),
and we get \( \mathbf{v}_n = \mathbf{a}_n - A_{n-1} {(A_{n-1}^T A_{n-1})}^{-1}A_{n-1}^T \mathbf{a}_n \).
\\

Remember that the matrix \( A_{n-1} \) is composed of orthonormal vectors, so \( A_{n-1}^T A_{n-1} = I \), then we have

\[ 
    \mathbf{v}_n = \mathbf{a}_n - A_{n-1} A_{n-1}^T \mathbf{a}_n  = \mathbf{a}_n - \mathbf{e}_1^T \mathbf{a}_n \mathbf{e}_1 - \mathbf{e}_2^T \mathbf{a}_n \mathbf{e}_2 - \cdots - \mathbf{e}_{n-1}^T \mathbf{a}_n \mathbf{e}_{n-1}
\]

if we want to do the normalization at the end, we can rewrite the above formula as below:

\[
    \mathbf{v}_n = \mathbf{a}_n - B_{n-1}{(B_{n-1}^T B_{n-1})}^{-1}B_{n-1}^T \mathbf{a}_n  = \mathbf{a}_n - \frac{\mathbf{u}_1^T \mathbf{a}_n}{\mathbf{u}_1^T \mathbf{u}_1} \mathbf{u}_1 - \frac{\mathbf{u}_2^T \mathbf{a}_n}{\mathbf{u}_2^T \mathbf{u}_2} \mathbf{u}_2 - \cdots - \frac{\mathbf{u}_{n-1}^T \mathbf{a}_n}{ \mathbf{u}_{n-1}^T \mathbf{u}_{n-1}} \mathbf{u}_{n-1}
\]

Here \( \mathbf{u}_i / || \mathbf{u}_i || = \mathbf{e}_i \), and matrix \( B_{n-1} \) is the matrix composed of \( \mathbf{u}_1, \ldots, \mathbf{u}_{n-1} \).

\pagebreak

\begin{homeworkProblem}

If \( \mathbf{q}_1 \) and \( \mathbf{q}_2 \) are orthonormal vectors in \( \mathbf{R}^5 \),
what combination \emptyunderline\( \mathbf{q}_1 \) + \emptyunderline\( \mathbf{q}_2 \) is closest
to a given vector \( \mathbf{b} \)?
\\

\solution
\\

\(\underline{\mathbf{q}_1^T \mathbf{b}}\ \mathbf{q}_1 + \underline{\mathbf{q}_2^T \mathbf{b}}\ \mathbf{q}_2  \).

\end{homeworkProblem}

\begin{homeworkProblem}
    What multiple of \( \mathbf{a} = \begin{bmatrix}
        1 \\
        1
    \end{bmatrix} \) should be subtracted from \( \mathbf{b} = \begin{bmatrix}
        4   \\
        0
    \end{bmatrix} \)
    to make the result \( \mathbf{B} \) orthogonal to \( \mathbf{a} \)? Sketch a figure
    to show \( \mathbf{a}, \mathbf{b}\ \text{and}\ \mathbf{B} \).
    \\

    Complete the Gram-Schmidt process for this problem by computing
    \( \mathbf{q}_1 = \mathbf{a} / || \mathbf{a} || \) and \( \mathbf{q}_2  = \mathbf{B} / || \mathbf{B} || \)
    and factoring into \( QR \):

    \[
        \begin{bmatrix}
            1   &   4   \\
            1   &   0
        \end{bmatrix}
        =
        \begin{bmatrix}
            \mathbf{q}_1 & \mathbf{q}_2
        \end{bmatrix}
        \begin{bmatrix}
            || \mathbf{a} ||  & ? \\
            || 0    ||  & || \mathbf{B} ||
        \end{bmatrix}
    \]

    \solution
    \\
    
    As shown at the beginning of this homework, we can get \( \mathbf{B} \) as below:

    \[
        \mathbf{B} = \mathbf{b} - \frac{\mathbf{a}^T \mathbf{b}}{\mathbf{a}^T \mathbf{a}} \mathbf{a} = \begin{bmatrix}
            2 \\
            -2
        \end{bmatrix}
    \]

    The multiple of \( \mathbf{a} \) that should be subtracted from \( \mathbf{b} \) is 2.

    The picture is shown below:

\[
    \begin{tikzpicture}[->,thick]
        \draw[color=black] (0, 0) -- (1, 1);
        \draw[color=black] (0,0) -- (4,0);
        \draw[color=black] (0,0) -- (2,-2);

            \node[above right] at (1, 1) {$\mathbf{a}$};
    \node[right] at (4, 0) {$\mathbf{b}$};
    \node[below right] at (2, -2) {$\mathbf{B}$};
    \end{tikzpicture}
\]

After computation we get \( \mathbf{q}_1 = \begin{bmatrix}
    1 / \sqrt{2} \\
    1 / \sqrt{2}
\end{bmatrix} \)
,
\( \mathbf{q}_2 = \begin{bmatrix}
    1 / \sqrt{2} \\
    -1 / \sqrt{2}
    \end{bmatrix}
\), then we have \( QR \) decomposition of the matrix \( A \):
\\

\[
    A = \begin{bmatrix}
        1   &   4   \\
        1   &   0
    \end{bmatrix} = \begin{bmatrix}
        1 / \sqrt{2} & 1 / \sqrt{2} \\
        1 / \sqrt{2} & -1 / \sqrt{2}
    \end{bmatrix}
    \begin{bmatrix}
        \sqrt{2}    &   2\sqrt{2}   \\
        0   &   2\sqrt{2}
    \end{bmatrix}
\]
\end{homeworkProblem}

\begin{homeworkProblem}
    Find \( \mathbf{q}_1, \mathbf{q}_2, \mathbf{q}_3 \) (orthonormal) as combinations of \( \mathbf{a}, \mathbf{b}, \mathbf{c} \)
    (independent columns). Then write \( A \) as \( QR \):
    \[
        A = \begin{bmatrix}
            1 & 2 & 4   \\
            0 & 0 & 5   \\
            0 & 3 & 6
        \end{bmatrix}
    \]

    \solution
    \\

    \( \mathbf{q}_1, \mathbf{q}_2, \mathbf{q}_3 \) are shown below:

    \[
        \begin{split}
            \mathbf{q}_1 =& \mathbf{a} = \begin{bmatrix}
                1 & 0 & 0
            \end{bmatrix}^T \\
            \mathbf{q}_2 =& (\mathbf{b} - \mathbf{q}_1^T \mathbf{b} \mathbf{q}_1) / || \mathbf{b} - \mathbf{q}_1^T \mathbf{b} \mathbf{q}_1 || = \begin{bmatrix} 0 & 0 & 1 \end{bmatrix}^T \\
            \mathbf{q}_3 =& (\mathbf{c} - \mathbf{q}_1^T \mathbf{c} \mathbf{q}_1 - \mathbf{q}_2^T \mathbf{c} \mathbf{q}_2) / || \mathbf{c} - \mathbf{q}_1^T \mathbf{c} \mathbf{q}_1 - \mathbf{q}_2^T \mathbf{c} \mathbf{q}_2 || = \begin{bmatrix} 0 & 1 & 0 \end{bmatrix}^T
        \end{split}
    \]

and then we can get the \( QR \) decomposition of the matrix \( A \):

\[
        A = \begin{bmatrix}
            1 & 2 & 4   \\
            0 & 0 & 5   \\
            0 & 3 & 6
        \end{bmatrix} = \begin{bmatrix}
            1 & 0 & 0   \\
            0 & 0 & 1   \\
            0 & 1 & 0
        \end{bmatrix}
        \begin{bmatrix}
            1 & 2 & 4   \\
            0 & 3 & 6   \\
            0 & 0 & 5
        \end{bmatrix}
\]

\end{homeworkProblem}

\begin{homeworkProblem}
    Choose \( c \) so that \( Q \) is an orthogonal matrix:
   
    \[
        Q = c\begin{bmatrix}
            1 & -1 & -1 & -1    \\
            -1 & 1 & -1 & -1    \\
            -1 & -1 & 1 & -1    \\
            -1 & -1 & -1 & 1
        \end{bmatrix}
    \]

    Project \( \mathbf{b} \) = (1, 1, 1, 1) onto the first column. Then
    project \( \mathbf{b} \) onto the plane of the first two columns.
    \\

    \solution

    \begin{enumerate}
        \item \( c = 1 / 2 \).
        \item The projection of \( \mathbf{b} \) onto the first column \( \mathbf{p} = \begin{bmatrix}
            -1/2 & 1/2 & 1/2 & 1/2
        \end{bmatrix}^T \).
        \item The projection of \( \mathbf{b} \) onto the first two columns is \( \mathbf{p} = \begin{bmatrix} 
            0 & 0 & 1 & 1
        \end{bmatrix}^T \).
    \end{enumerate}
\end{homeworkProblem}

\begin{homeworkProblem}
    If you add row 1 = \( \begin{bmatrix}
        a & b & c
    \end{bmatrix} \) to row 2 \( \begin{bmatrix}
        p & q & r
    \end{bmatrix} \)
    to get \( \begin{bmatrix}
        p + a & q + b & r + c   \\
    \end{bmatrix} \)
    in row 2, show from formula{(1)} for \( \det A \) that the 3 by 3
    determinant does not change. Here is another approach to the rule for adding
    two rows:

    \[
        \det \begin{bmatrix}
            \text{row 1}    \\
            \textbf{row1 + row2}    \\
            \text{row 3}
        \end{bmatrix}
        =
        \det \begin{bmatrix}
            \text{row 1}    \\
            \textbf{row 1}    \\
            \text{row 3}
        \end{bmatrix}
        +
        \det \begin{bmatrix}
            \text{row 1}    \\
            \textbf{row 2}    \\
            \text{row 3}
        \end{bmatrix}
        =
        \mathbf{0}
        +
        \det \begin{bmatrix}
            \text{row 1}    \\
            \text{row 2}    \\
            \text{row 3}
        \end{bmatrix}
    \]

    \solution


\end{homeworkProblem}

\begin{homeworkProblem}
    Do these martices have determinant 0,1,2, or 3?
    
    \[
        A = \begin{bmatrix}
            0 & 0 & 1   \\
            1 & 0 & 0   \\
            0 & 1 & 0
        \end{bmatrix}
        \ \ \ 
        B = \begin{bmatrix}
            0 & 1 & 1   \\
            1 & 0 & 1   \\
            1 & 1 & 0
        \end{bmatrix}
        \ \ \ 
        C = \begin{bmatrix}
            1 & 1 & 1   \\
            1 & 1 & 1   \\
            1 & 1 & 1
        \end{bmatrix}
        \ \ \ 
        D = \begin{bmatrix}
            1 & 1 & 1   \\
            0 & 1 & 0   \\
            1 & 1 & 1
        \end{bmatrix}
    \]

    \solution

    \begin{enumerate}
        \item \( \det A = 1 \).
        \item \( \det B = 2 \).
        \item \( \det C = 0 \).
        \item \( \det D = 0 \).
    \end{enumerate}
\end{homeworkProblem}

\begin{homeworkProblem}
    Show that \( \det A = 0 \), regardless of the five numbers marked by \( x \)'s:

    \[
        A = \begin{bmatrix}
            x & x & x   \\
            0 & 0 & x   \\
            0 & 0 & x
        \end{bmatrix}
    \]
    \\

    What are the cofactors of row 1? What is the rank of \( A \)? 
    What are the 6 terms in \( \det A \)?
    \\

    \solution

    \begin{enumerate}
        \item The corresponding cofactor matrix is: \( \begin{bmatrix}
            0 & 0 & 0   \\
            -x^2 & x^2 & 0   \\
            x^2 & -x^2 & 0 
        \end{bmatrix} \).
        \item If \( x \neq 0 \), the rank of \( A \) is 2, otherwise the rank of \( A \) is 0.
        \item The 6 terms in \( \det A \) are all zero.
    \end{enumerate}
\end{homeworkProblem}

\begin{homeworkProblem}
    Quick proof of Cramer's rule. The determinant is a linear function
    of column 1. It is zero if two columns are equal. When \( \mathbf{b} = A \mathbf{x} = x_1\mathbf{a}_1 + x_2\mathbf{a}_2 + x_3\mathbf{a}_3\) goes into the first column of
    \( A \), we have the matrix \( B_1 \) and Cramer's Rule \( x_1 = \det B_1 / \det A \):

    \[
        | \mathbf{b}\ \ \mathbf{a}_2\ \ \mathbf{a}_3 | = |x_1\mathbf{a}_1 + x_2\mathbf{a}_2 + x_3\mathbf{a}_3\ \ \mathbf{a}_2\ \ \mathbf{a}_3|
        = x_1|\mathbf{a}_1\ \ \mathbf{a}_2\ \ \mathbf{a}_3| = x_1 \det A.
    \]

    What steps lead to the middle equation?
    \\

    \solution

    \[  
        \begin{split}
            | \mathbf{b}\ \ \mathbf{a}_2\ \ \mathbf{a}_3 | =& |x_1\mathbf{a}_1 + x_2\mathbf{a}_2 + x_3\mathbf{a}_3\ \ \mathbf{a}_2\ \ \mathbf{a}_3|  \\
            =& |x_1 \mathbf{a}_1\ \ \mathbf{a}_2\ \ \mathbf{a}_3| + |x_2\mathbf{a}_2\ \ \mathbf{a}_2\ \ \mathbf{a}_3| + |x_3\mathbf{a}_3\ \ \mathbf{a}_2\ \ \mathbf{a}_3|   \\
            =& x_1|\mathbf{a}_1\ \ \mathbf{a}_2\ \ \mathbf{a}_3| + x_2|\mathbf{a}_2\ \ \mathbf{a}_2\ \ \mathbf{a}_3| + x_3|\mathbf{a}_3\ \ \mathbf{a}_2\ \ \mathbf{a}_3|
        \end{split}
    \]

    Using the fact that the determinant is zero if two columns are equal, we then have
    \( |\mathbf{b}\ \ \mathbf{a}_2\ \ \mathbf{a}_3| = x_1\det A\).

\end{homeworkProblem}

\begin{homeworkProblem}
    (prize for the max determinant) If a 3 by 3 matrix has entries 1,2,3,4, \ldots ,9, what is the
    maximum determinant? I would use a computer to decide. This problem does not seem easy.
    \\

    \solution

    \[
        \begin{bmatrix}
            1 & 5 & 7   \\
            4 & 9 & 2   \\
            8 & 3 & 6
        \end{bmatrix}
    \]

    The code is shown below:

\begin{lstlisting}[language=Python]
import numpy
import itertools

max = 0

for p in itertools.permutations(range(1,10)):
    matrix = [p[0:3], p[3:6], p[6:9]]
    temp = abs(numpy.linalg.det(matrix))
    if temp > max:
        max = temp
        solution = matrix

print(solution)
\end{lstlisting}

\end{homeworkProblem}

\end{document}