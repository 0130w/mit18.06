\documentclass{article}

\usepackage{fancyhdr}
\usepackage{extramarks}
\usepackage{amsmath}
\usepackage{amsthm}
\usepackage{amsfonts}
\usepackage{tikz}
\usepackage[plain]{algorithm}
\usepackage{algpseudocode}

\usetikzlibrary{automata,positioning}

%
% Basic Document Settings
%

\topmargin=-0.45in
\evensidemargin=0in
\oddsidemargin=0in
\textwidth=6.5in
\textheight=9.0in
\headsep=0.25in

\linespread{1.1}

\pagestyle{fancy}
\lhead{\hmwkAuthorName}
\chead{\hmwkClass\ (\hmwkClassInstructor): \hmwkTitle}
\rhead{\firstxmark}
\lfoot{\lastxmark}
\cfoot{\thepage}

\renewcommand\headrulewidth{0.4pt}
\renewcommand\footrulewidth{0.4pt}

\setlength\parindent{0pt}

%
% Create Problem Sections
%

\newcommand{\enterProblemHeader}[1]{
    \nobreak\extramarks{}{Problem \arabic{#1} continued on next page\ldots}\nobreak{}
    \nobreak\extramarks{Problem \arabic{#1} (continued)}{Problem \arabic{#1} continued on next page\ldots}\nobreak{}
}

\newcommand{\exitProblemHeader}[1]{
    \nobreak\extramarks{Problem \arabic{#1} (continued)}{Problem \arabic{#1} continued on next page\ldots}\nobreak{}
    \stepcounter{#1}
    \nobreak\extramarks{Problem \arabic{#1}}{}\nobreak{}
}

\setcounter{secnumdepth}{0}
\newcounter{partCounter}
\newcounter{homeworkProblemCounter}
\setcounter{homeworkProblemCounter}{1}
\nobreak\extramarks{Problem \arabic{homeworkProblemCounter}}{}\nobreak{}

%
% Homework Problem Environment
%
% This environment takes an optional argument. When given, it will adjust the
% problem counter. This is useful for when the problems given for your
% assignment aren't sequential. See the last 3 problems of this template for an
% example.
%
\newenvironment{homeworkProblem}[1][-1]{
    \ifnum#1>0
        \setcounter{homeworkProblemCounter}{#1}
    \fi
    \section{Problem \arabic{homeworkProblemCounter}}
    \setcounter{partCounter}{1}
    \enterProblemHeader{homeworkProblemCounter}
}{
    \exitProblemHeader{homeworkProblemCounter}
}

%
% Homework Details
%   - Title
%   - Due date
%   - Class
%   - Section/Time
%   - Instructor
%   - Author
%

\newcommand{\hmwkTitle}{Homework\ \#1}
\newcommand{\hmwkDueDate}{February 24, 2024}
\newcommand{\hmwkClass}{Linear Algebra}
\newcommand{\hmwkClassInstructor}{Gilbert Strang}
\newcommand{\hmwkAuthorName}{\textbf{0130}}

%
% Title Page
%

\title{
    \vspace{2in}
    \textmd{\textbf{\hmwkClass:\ \hmwkTitle}}\\
    \vspace{0.1in}\large{\textit{\hmwkClassInstructor}}
    \vspace{3in}
}

\author{\hmwkAuthorName}
\date{}

\renewcommand{\part}[1]{\textbf{\large Part \Alph{partCounter}}\stepcounter{partCounter}\\}

%
% Various Helper Commands
%

% Useful for algorithms
\newcommand{\alg}[1]{\textsc{\bfseries \footnotesize #1}}

% For derivatives
\newcommand{\deriv}[1]{\frac{\mathrm{d}}{\mathrm{d}x} (#1)}

% For partial derivatives
\newcommand{\pderiv}[2]{\frac{\partial}{\partial #1} (#2)}

% Integral dx
\newcommand{\dx}{\mathrm{d}x}

% Alias for the Solution section header
\newcommand{\solution}{\textbf{\large Solution}}

% Probability commands: Expectation, Variance, Covariance, Bias
\newcommand{\E}{\mathrm{E}}
\newcommand{\Var}{\mathrm{Var}}
\newcommand{\Cov}{\mathrm{Cov}}
\newcommand{\Bias}{\mathrm{Bias}}

\begin{document}

\maketitle

\pagebreak

\begin{homeworkProblem}
    How could you decide if the vectors \( \mathbf{u} = (1,1,0)\) and \( 
    \mathbf{v} = (0,1,1) \) and \( \mathbf{w} = (a,b,c) \) are linearly
    independent or dependent?
    \\

    \solution

    If \( a + c - b \neq 0 \), then these vectors are linearly independent,
    otherwise they are linearly dependent.

\end{homeworkProblem}

\begin{homeworkProblem}
    How many corners \( (\pm 1, \pm 1, \pm 1, \pm 1) \) does a cube of side 2
    have in 4 dimensions? What is its volume? How many 3D faces? How many edges?
    Find one edge.
    \\

    \solution
    
    \begin{enumerate}
        \item The cube has \( 16 \) corners.
        \item The volume of the cube is \( 16 \).
        \item There are \( 8 \) 3D faces.
        \item There are \( 32 \) edges.
        \item \( \left\{ (-1, -1, -1, -1), (-1, -1, -1, 1) \right\} \) is one edge.
    \end{enumerate}
\end{homeworkProblem}

\begin{homeworkProblem}
    The \textbf{triangule inequality} says: (length of \( \mathbf{v + w} \)) \( \leq \) (length
     of \( \mathbf{v} \)) + (length of \( \mathbf{w} \)).
    \begin{enumerate}
        \item Show that \( ||\mathbf{v + w} ||^2 = ||\mathbf{v}||^2 + 2\mathbf{v}\cdot \mathbf{w}
        + ||\mathbf{w}||^2\).
        \item Increase that \( \mathbf{v \cdot w}\) to ||\( \mathbf{v}\)||\ ||\( \mathbf{w}\)||
            to show that ||\textbf{side\ 3}|| cannot exceed ||\textbf{side\ 1}|| + ||\textbf{side\ 2}||:
    \end{enumerate}

    \solution

    \begin{enumerate}
        \item \( \mathbf{||v + w||^2 = (v + w) \cdot (v+w) = v^2 + }2\mathbf{v\cdot w + w^2 = ||v||^2 + } 
        2\mathbf{v\cdot w + ||w||^2}\)
        \item \( \mathbf{||side\ 3|| = ||side\ 1 - side\ 2|| \leq ||side\ 1|| + ||-side\ 2|| = ||side\ 1|| + ||side\ 2||} \)
    \end{enumerate}
\end{homeworkProblem}

\begin{homeworkProblem}
    Which numbers \( q \) would leave \( A \) with two 
    independent columns ?
    \[  A = 
        \begin{bmatrix}
            1 & 0 & 2   \\
            3 & 1 & 9   \\
            5 & 0 & q
        \end{bmatrix}
        ,
        A = 
        \begin{bmatrix}
            1 & 4 & 7   \\
            2 & 5 & 8   \\
            3 & 6 & q
        \end{bmatrix}
        ,
        A =
        \begin{bmatrix}
            1 & 1 & 2   \\
            2 & 2 & 4   \\
            0 & 0 & q
        \end{bmatrix}
    \]

    \solution

    \begin{enumerate}
        \item \( q = 10\).
        \item \( q = 9\).
        \item \( q \neq 0 \).
    \end{enumerate}
\end{homeworkProblem}

\begin{homeworkProblem}
    If \( (a,b) \) is a multiple of \( (c,d) \) with \( abcd \neq 0\),
    show that \( (a,c) \) is a multiple of \( (b,d) \). This is
    surprisingly important; two columns are failing on one line.
    You could use numbers first to see how \( a, b, c, d\) are
    related. The question will lead to:
    \\
    \[
    \text{If}
    \begin{bmatrix}
        a & b   \\
        c & d
    \end{bmatrix}
    \text{has dependent rows, then it also has dependent
    columns.}
    \]

    \solution
    \\

    Suppose \( (a,c) \) is \( k \) times \( (b,d) \), then we have \( \frac{a}{b}
     = \frac{kc}{kd} = \frac{c}{d} \).

\end{homeworkProblem}

\begin{homeworkProblem}
    Why is it impossible for a matrix A with 7 columns and 4 rows to
    have 5 independent columns? This is not a trivial or useless question.
    \\

    \solution
    \\

    This question is equivalent to asking the rank of a matrix is less than
    or equal to the lesser of them. Recall the process of elimination, if 
    we have 5 independent columns, then we should have 5 pivot, which conflicts
     to the fact that we only have 4 rows.

\end{homeworkProblem}

\begin{homeworkProblem}
    Going from left to right, put each column of \( A \) into the
    matrix \( C \) if that column is not a combination of earlier columns:

    \[
        A = 
        \begin{bmatrix}
            2 & -2 & 1 & 6 & 0  \\
            1 & -1 & 0 & 2 & 0  \\
            3 & -3 & 0 & 6 & 1
        \end{bmatrix}
        \ \ \ 
        C = 
        \begin{bmatrix}
            2 &  &  \\
            1 &  &  \\
            3 &  &  
        \end{bmatrix}
    \]

    Find \( R \) in Problem 6 so that \( A = CR \).
    If your \( C \) has \( r \) columns, then \( R \)
    has \( r \) rows. The 5 columns of \( R \) tell how
    to produce the 5 columns of \( A \) from the columns
    in \( C \).
    \\

    \solution

    \[
        R = 
        \begin{bmatrix}
            -1 & -1 & 0 & 2 & 0 \\
            0 & 0 & 1 & 2 & 0   \\
            0 & 0 & 0 & 0 & 1
        \end{bmatrix}
    \]
\end{homeworkProblem}

\begin{homeworkProblem}
    Complete these 2 by 2 matrices to meet the requirements printed
    underneath:
    \[
\begin{array}{cccc}
    \begin{bmatrix}
        3 & 6 \\
        5 &
    \end{bmatrix} &
    \begin{bmatrix}
        6 & 7 \\
        6 &
    \end{bmatrix} &
    \begin{bmatrix}
        3 &   \\
        3 & 6
    \end{bmatrix} &
    \begin{bmatrix}
        3 & 4 \\
          & -3
    \end{bmatrix} \\
    \text{rank one} & \text{orthogonal columns} & \text{rank 2} & A^2 = I
\end{array}
\]

\solution

\[
    \begin{bmatrix} 
        3 & 6 \\ 5 & 10
        \end{bmatrix}
        ,
        \begin{bmatrix}
        6 & 7 \\ 7 & -6
        \end{bmatrix}
        ,
        \begin{bmatrix}
        2 & 3 \\ 3 & 6
        \end{bmatrix}
        ,
        \begin{bmatrix}
        3 & 4 \\ -2 & -3
        \end{bmatrix}
\]

\end{homeworkProblem}

\begin{homeworkProblem}
    True or false, with a reason:
    \begin{enumerate}
        \item  If 3 by 3 matrices \( A \) and \( B \) have rank 1,
         then \( AB \) will always have rank 1.
        \item If 3 by 3 matrices \( A \) and \( B \) have rank 3,
         then \( AB \) will always have rank 3.
        \item  Suppose \( AB = BA \) for every 2 by 2 matrix \( B \). Then
        \( A = \begin{bmatrix}
            c & 0   \\
            0 & c
         \end{bmatrix}
            = cI
        \) for some number \( c \).
        Only those matrices \( A = cI \) commute with every \( B \).
        \\

        \solution

        \begin{enumerate}
            \item No.consider the following matrices
                \[
                    A = 
                    \begin{bmatrix}
                        1 & 0 & 0   \\
                        0 & 0 & 0   \\
                        0 & 0 & 0
                    \end{bmatrix}
                    ,
                    B = 
                    \begin{bmatrix}
                        0 & 0 & 0   \\
                        0 & 1 & 0   \\
                        0 & 0 & 0
                    \end{bmatrix}
                \]
            \item Yes, rank (\( AB \)) = rank (\( A \)) = 3
                since rank (\( B \)) = 3.
            \item Omitted.
        \end{enumerate}
    \end{enumerate}
\end{homeworkProblem}

\begin{homeworkProblem}
    How many small multiplications for \( (AB)C \) and \( A(BC) \) if those
    matrices have sizes ABC = \( (4 \times 3) (3 \times 2) (2 \times 1)\)? The two counts are different.
    \\

    \solution
    \\
    32;24

\end{homeworkProblem}

\end{document}