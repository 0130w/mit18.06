\documentclass{article}

\usepackage{fancyhdr}
\usepackage{extramarks}
\usepackage{boondox-cal}
\usepackage{amsmath}
\usepackage{amsthm}
\usepackage{amsfonts}
\usepackage{tikz}
\usepackage[plain]{algorithm}
\usepackage{listings}
\usepackage{algpseudocode}

\usetikzlibrary{automata,positioning, arrows.meta}

%
% Basic Document Settings
%

\topmargin=-0.45in
\evensidemargin=0in
\oddsidemargin=0in
\textwidth=6.5in
\textheight=9.0in
\headsep=0.25in

\linespread{1.1}

\pagestyle{fancy}
\lhead{\hmwkAuthorName}
\chead{\hmwkClass\ (\hmwkClassInstructor): \hmwkTitle}
\rhead{\firstxmark}
\lfoot{\lastxmark}
\cfoot{\thepage}

\renewcommand\headrulewidth{0.4pt}
\renewcommand\footrulewidth{0.4pt}

\setlength\parindent{0pt}

%
% Create Problem Sections
%

\newcommand{\enterProblemHeader}[1]{
    \nobreak\extramarks{}{Problem \arabic{#1} continued on next page\ldots}\nobreak{}
    \nobreak\extramarks{Problem \arabic{#1} (continued)}{Problem \arabic{#1} continued on next page\ldots}\nobreak{}
}

\newcommand{\exitProblemHeader}[1]{
    \nobreak\extramarks{Problem \arabic{#1} (continued)}{Problem \arabic{#1} continued on next page\ldots}\nobreak{}
    \stepcounter{#1}
    \nobreak\extramarks{Problem \arabic{#1}}{}\nobreak{}
}

\setcounter{secnumdepth}{0}
\newcounter{partCounter}
\newcounter{homeworkProblemCounter}
\setcounter{homeworkProblemCounter}{1}
\nobreak\extramarks{Problem \arabic{homeworkProblemCounter}}{}\nobreak{}

%
% Homework Problem Environment
%
% This environment takes an optional argument. When given, it will adjust the
% problem counter. This is useful for when the problems given for your
% assignment aren't sequential. See the last 3 problems of this template for an
% example.
%
\newenvironment{homeworkProblem}[1][-1]{
    \ifnum#1>0
        \setcounter{homeworkProblemCounter}{#1}
    \fi
    \section{Problem \arabic{homeworkProblemCounter}}
    \setcounter{partCounter}{1}
    \enterProblemHeader{homeworkProblemCounter}
}{
    \exitProblemHeader{homeworkProblemCounter}
}

%
% Homework Details
%   - Title
%   - Due date
%   - Class
%   - Section/Time
%   - Instructor
%   - Author
%

\newcommand{\hmwkTitle}{Homework\ \#8}
\newcommand{\hmwkDueDate}{March 10, 2024}
\newcommand{\hmwkClass}{Linear Algebra}
\newcommand{\hmwkClassInstructor}{Gilbert Strang}
\newcommand{\hmwkAuthorName}{\textbf{0130}}

%
% Title Page
%

\title{
    \vspace{2in}
    \textmd{\textbf{\hmwkClass:\ \hmwkTitle}}\\
    \vspace{0.1in}\large{\textit{\hmwkClassInstructor}}
    \vspace{3in}
}

\author{\hmwkAuthorName}
\date{}

\renewcommand{\part}[1]{\textbf{\large Part \Alph{partCounter}}\stepcounter{partCounter}\\}

%
% Various Helper Commands
%

% Useful for algorithms
\newcommand{\alg}[1]{\textsc{\bfseries \footnotesize #1}}

% For derivatives
\newcommand{\deriv}[1]{\frac{\mathrm{d}}{\mathrm{d}x} (#1)}

% For partial derivatives
\newcommand{\pderiv}[2]{\frac{\partial}{\partial #1} (#2)}

% Integral dx
\newcommand{\dx}{\mathrm{d}x}

% Alias for the Solution section header
\newcommand{\solution}{\textbf{\large Solution}}

% Probability commands: Expectation, Variance, Covariance, Bias
\newcommand{\E}{\mathrm{E}}
\newcommand{\Var}{\mathrm{Var}}
\newcommand{\Cov}{\mathrm{Cov}}
\newcommand{\Bias}{\mathrm{Bias}}

% empty underline
\newcommand{\emptyunderline}{\underline{\ \ \ \ \ \ }}

\begin{document}

\maketitle

\pagebreak

\begin{homeworkProblem}
Find the eigenvalues and singular values of this 2 by 2 matrix \( A \).

\[
    A = \begin{bmatrix}
        2 & 1 \\ 4 & 2
    \end{bmatrix}
    \text{ with }
    A^T A = \begin{bmatrix}
        20 & 10 \\ 10 & 5
    \end{bmatrix}
    \text{ and }
    A A^T = \begin{bmatrix}
        5 & 10 \\ 10 & 20
    \end{bmatrix}
\]

The eigenvectors \( (1, 2) \) and \( (1, -2) \) of \( A \) are not orthogonal.
How do you know the eigenvectors \( \mathbf{v}_1, \mathbf{v}_2 \) of \( A^T A \) will be orthogonal?
Notice that \( A^T A \) and \( A A^T \) have the same eigenvalues \( \lambda_1 = 25 \text{ and } \lambda_2 = 0 \).
\\

\solution

\begin{enumerate}
    \item The eigenvalues of \( A \) are 0 and 4.
    \item The singular values of \( A \) are 5 and 0.
    \item Because the matrix \( A^T A \) is symmetric, its eigenvectors will be orthogonal.
\end{enumerate}

\end{homeworkProblem}

\begin{homeworkProblem}
    Find \( A^T A \) and \( A A^T \) and the singular vectors \( \mathbf{v}_1, \mathbf{v}_2, \mathbf{u}_1, \mathbf{u}_2 \) for \( A \):
    \[
        A = \begin{bmatrix}
            0 & 1 & 0   \\
            0 & 0 & 8   \\
            0 & 0 & 0
        \end{bmatrix}
        \text{  has rank }
        r = 2.
        \text{  The eigenvalues are }
        0, 0, 0 
    \]
    Check the equations \( A\mathbf{v}_1 = \sigma_1 \mathbf{u}_1 \) and \( A\mathbf{v}_2 = \sigma_2 \mathbf{u}_2 \)
    and \( A = \sigma_1 \mathbf{u_1}\mathbf{v}_1^T + \sigma_2 \mathbf{u}_2 \mathbf{v}_2^T \). If you remove row 3 by \( A \)
    (all zeros), show that \( \sigma_1 \) and \( \sigma_2 \) don't change.
    \\

    \solution

    \begin{enumerate}
        \item \( A^T A = \begin{bmatrix}
            0 & 0 & 0 \\
            0 & 1 & 0 \\
            0 & 0 & 64
        \end{bmatrix} 
        \), \( A A^T = \begin{bmatrix}
            1 & 0 & 0 \\
            0 & 64 & 0 \\
            0 & 0 & 0
        \end{bmatrix} \), the singular values of \( A \) are 1 and 8,
        the singular vectors are \( \mathbf{v}_1 = \begin{bmatrix}
            0 & 1 & 0
        \end{bmatrix}^T, \mathbf{v}_2 = \begin{bmatrix}
            0 & 0 & 1
        \end{bmatrix}^T, \mathbf{u}_1 = \begin{bmatrix}
            1 & 0 & 0
        \end{bmatrix}^T, \mathbf{u}_2 = \begin{bmatrix}
            0 & 1 & 0
        \end{bmatrix}^T \).
        \item Omitted.
        \item If remove row 3 by \( A \), then \( A = \begin{bmatrix}
            0 & 1 & 0 \\
            0 & 0 & 8
        \end{bmatrix} \), we get \( A^T A = \begin{bmatrix}
            0 & 0 & 0 \\
            0 & 1 & 0 \\
            0 & 0 & 64
        \end{bmatrix} \).
    \end{enumerate}
\end{homeworkProblem}

\begin{homeworkProblem}
    If \( (A^T A)\mathbf{v} = \sigma^2 \mathbf{v} \), multiply by \( A \). Move the 
    parentheses to get \( (A A^T)A \mathbf{v} = \sigma^2 (A\mathbf{v}) \).

    \( \textbf{If } \mathbf{v} \textbf{ is an eigenvector of } \mathbf{A^T A} \textbf{, then \emptyunderline is an eigenvector of } \mathbf{A A^T}\).
    \\

    \solution
    \\

    \( A\mathbf{v} \) is an eigenvector of \( A A^T \).
\end{homeworkProblem}

\begin{homeworkProblem}
    If \( A = Q \) is an orthogonal matrix, why does every singular value of \( Q \) equal 1?
    \\

    \solution
    \\

    Since \( Q \) is an orthogonal matrix, we have \( Q^T Q = I \), then we know:
    \[
        Q^T Q \mathbf{x} = \mathbf{x} = \lambda \mathbf{x} \Rightarrow \lambda = 1
    \]
\end{homeworkProblem}

\begin{homeworkProblem}
    \begin{enumerate}
        \item Why is the trace of \( A^T A \) equal to the sum of all \( a_{ij}^2? \)
        \item For every rank-one matrix, why is \( \sigma_1^2 = \text{sum of all } a_{ij}^2? \)
    \end{enumerate}

    \solution

    \begin{enumerate}
        \item \( \text{trace} (A^T A) = \sum_{i} {(A^T A)}_{ii} = \sum_{i}\sum_{j}A^T_{ij} A_{ji} = \sum_{i,j} a_{ij}^2\)
        \item For rank-one matrix \( A \), the number of singular value is only one, and the square of this singular value \( \sigma_1 \) is the only
        non-zero eigenvalue of \( A^T A \), so \( \sigma_1^2 = \text{trace} (A^T A) = \text{ sum of all } a_{ij}^2 \).
    \end{enumerate}
\end{homeworkProblem}

\begin{homeworkProblem}
    Suppose \( A_0 \) holds these 2 measurements of 5 samples:
    \[
        A_0 = \begin{bmatrix}
            5 & 4 & 3 & 2 & 1   \\
            -1 & 1 & 0 & 1 & -1
        \end{bmatrix}
    \]

    Find the average of each row and subtract it to produce the
    centered matrix \( A \). Compute the sample covariance matrix \( S = A A^T / (n-1) \) and 
    find its eigenvalues \( \lambda_1 \) and \( \lambda_2 \). What line
    through the origin is closest to the 5 samples in columns of \( A \)?
    \\

    \solution
    \\
    
    Centered matrix \( A \) = 
    \( \begin{bmatrix}
        2 & 1 & 0 & -1 & -2 \\
        -1 & 1 & 0 & 1 & -1
    \end{bmatrix} \), sample covariance matrix \( S \) = \( \begin{bmatrix}
        5/2 & 0 \\
        0 & 1
    \end{bmatrix} \), the eigenvalues of \( S \) are \( \lambda_1 = 5/2, \lambda_2 = 1 \),
    the corresponding eigenvectors are \( \begin{bmatrix}
        1 \\ 0
    \end{bmatrix}, \begin{bmatrix}
        0 \\ 1
    \end{bmatrix} \), then we know the closest line through the origin is \( y = 0 \).
\end{homeworkProblem}

\end{document}