\documentclass{article}

\usepackage{fancyhdr}
\usepackage{extramarks}
\usepackage{boondox-cal}
\usepackage{amsmath}
\usepackage{amsthm}
\usepackage{amsfonts}
\usepackage{tikz}
\usepackage[plain]{algorithm}
\usepackage{listings}
\usepackage{algpseudocode}

\usetikzlibrary{automata,positioning, arrows.meta}

%
% Basic Document Settings
%

\topmargin=-0.45in
\evensidemargin=0in
\oddsidemargin=0in
\textwidth=6.5in
\textheight=9.0in
\headsep=0.25in

\linespread{1.1}

\pagestyle{fancy}
\lhead{\hmwkAuthorName}
\chead{\hmwkClass\ (\hmwkClassInstructor): \hmwkTitle}
\rhead{\firstxmark}
\lfoot{\lastxmark}
\cfoot{\thepage}

\renewcommand\headrulewidth{0.4pt}
\renewcommand\footrulewidth{0.4pt}

\setlength\parindent{0pt}

%
% Create Problem Sections
%

\newcommand{\enterProblemHeader}[1]{
    \nobreak\extramarks{}{Problem \arabic{#1} continued on next page\ldots}\nobreak{}
    \nobreak\extramarks{Problem \arabic{#1} (continued)}{Problem \arabic{#1} continued on next page\ldots}\nobreak{}
}

\newcommand{\exitProblemHeader}[1]{
    \nobreak\extramarks{Problem \arabic{#1} (continued)}{Problem \arabic{#1} continued on next page\ldots}\nobreak{}
    \stepcounter{#1}
    \nobreak\extramarks{Problem \arabic{#1}}{}\nobreak{}
}

\setcounter{secnumdepth}{0}
\newcounter{partCounter}
\newcounter{homeworkProblemCounter}
\setcounter{homeworkProblemCounter}{1}
\nobreak\extramarks{Problem \arabic{homeworkProblemCounter}}{}\nobreak{}

%
% Homework Problem Environment
%
% This environment takes an optional argument. When given, it will adjust the
% problem counter. This is useful for when the problems given for your
% assignment aren't sequential. See the last 3 problems of this template for an
% example.
%
\newenvironment{homeworkProblem}[1][-1]{
    \ifnum#1>0
        \setcounter{homeworkProblemCounter}{#1}
    \fi
    \section{Problem \arabic{homeworkProblemCounter}}
    \setcounter{partCounter}{1}
    \enterProblemHeader{homeworkProblemCounter}
}{
    \exitProblemHeader{homeworkProblemCounter}
}

%
% Homework Details
%   - Title
%   - Due date
%   - Class
%   - Section/Time
%   - Instructor
%   - Author
%

\newcommand{\hmwkTitle}{Homework\ \#6}
\newcommand{\hmwkDueDate}{March 2, 2024}
\newcommand{\hmwkClass}{Linear Algebra}
\newcommand{\hmwkClassInstructor}{Gilbert Strang}
\newcommand{\hmwkAuthorName}{\textbf{0130}}

%
% Title Page
%

\title{
    \vspace{2in}
    \textmd{\textbf{\hmwkClass:\ \hmwkTitle}}\\
    \vspace{0.1in}\large{\textit{\hmwkClassInstructor}}
    \vspace{3in}
}

\author{\hmwkAuthorName}
\date{}

\renewcommand{\part}[1]{\textbf{\large Part \Alph{partCounter}}\stepcounter{partCounter}\\}

%
% Various Helper Commands
%

% Useful for algorithms
\newcommand{\alg}[1]{\textsc{\bfseries \footnotesize #1}}

% For derivatives
\newcommand{\deriv}[1]{\frac{\mathrm{d}}{\mathrm{d}x} (#1)}

% For partial derivatives
\newcommand{\pderiv}[2]{\frac{\partial}{\partial #1} (#2)}

% Integral dx
\newcommand{\dx}{\mathrm{d}x}

% Alias for the Solution section header
\newcommand{\solution}{\textbf{\large Solution}}

% Probability commands: Expectation, Variance, Covariance, Bias
\newcommand{\E}{\mathrm{E}}
\newcommand{\Var}{\mathrm{Var}}
\newcommand{\Cov}{\mathrm{Cov}}
\newcommand{\Bias}{\mathrm{Bias}}

% empty underline
\newcommand{\emptyunderline}{\underline{\ \ \ \ \ \ }}

\begin{document}

\maketitle

\pagebreak

\begin{homeworkProblem}

    The example at the start of the chapter has powers of this matrix \( A \):
    \[
        A = \begin{bmatrix}
            .8 & .3 \\
            .2 & .7
        \end{bmatrix}
        \ \ \ 
        \text{and}
        A^2 = \begin{bmatrix}
            .70 & .45   \\
            .30 & .55
        \end{bmatrix}
        \ \ \ 
        \text{and}
        \ \ \ 
        A^\infty = \begin{bmatrix}
            .6 & .6 \\
            .4 & .4
        \end{bmatrix}
        .
    \]

    Find the eigenvalues of these matrices. All powers have the same eigenvectors.
    Show from \( A \) how a row exchange can produce different eigenvalues.
    \\

    \solution

    \begin{enumerate}
        \item The eigenvalues of \( A \) are: \( 1/2, 1 \).
        \item The eigenvalues of \( A^2 \) are: \( 1/4, 1 \).
        \item The eigenvalues of \( A^\infty \) of \( 0, 1 \).
    \end{enumerate}

    If we exchange two rows of \( A \), then the eigenvalues of the resulting matrix will be:
    \( -1/2, 1 \).
\end{homeworkProblem}

\begin{homeworkProblem}
    Find three eigenvectors for this matrix \( P \) (projection matrices have \( \lambda = 1 \) and 0):

    \[
        \textbf{Projection matrix}
        \ \ \ \ \ \ 
        P = \begin{bmatrix}
            .2 & .4 & 0 \\
            .4 & .8 & 0 \\
            0 & 0 & 1
        \end{bmatrix}
    \]

    If two eigenvectors share the same \( \lambda \), so do all their linear combinations. Find
    an eigenvector of \( P \) with no zero components.
    \\

    \solution
    \\

    The eigenvectors corresponding to the eigenvalue \( \lambda = 1 \) are: \( \begin{bmatrix}
        1 & 2 & 0
    \end{bmatrix}^T \) and \(
    \begin{bmatrix}
        0 & 0 & 1
    \end{bmatrix}^T \). The eigenvector corresponding to the eigenvalue \( \lambda = 0 \) is: \( \begin{bmatrix}
        -2 & 1 & 0
    \end{bmatrix}^T \).
    \\

    A eigenvector of \( P \) with no zero components is: \( \begin{bmatrix}
        1 & 2 & 1 
    \end{bmatrix}^T \).
    
\end{homeworkProblem}

\begin{homeworkProblem}
    A 3 by 3 matrix \( B \) is known to have eigenvalues 0,1,2. This information is enough to find
    three of these (give the answer where possible):

    \begin{enumerate}
        \item the rank of \( B \)
        \item the determinant of \( B^T B \)
        \item the eigenvalues of \( B^T B \)
        \item the eigenvalues of \( {(B^2 + I)}^{-1} \)
    \end{enumerate}

    \solution
    
    \begin{enumerate}
        \item The rank of \( B \) is 3.
        \item The determinant of \( B^T B \) is \( 0 \).
        \item The eigenvalues of \( B^T B \) are unknown.
        \item The eigenvalues of \( {(B^2 + I)}^{-1} \) are 1, 1/2, 1/5.
    \end{enumerate}
\end{homeworkProblem}

\begin{homeworkProblem}
    The matrix is singular with rank one. Find three \( \lambda \)'s and three eigenvectors:

    \[
        A = \begin{bmatrix}
            1 \\ 2 \\ 1 \\
        \end{bmatrix}
        \begin{bmatrix}
            2 & 1 & 2
        \end{bmatrix}
        =
        \begin{bmatrix}
            2 & 1 & 2   \\
            4 & 2 & 4   \\
            2 & 1 & 2
        \end{bmatrix}
    \]

    \solution
    \\

    Three eigenvalues are 0, 0, 6. The corresponding eigenvectors are:
    \( \begin{bmatrix}
        -1 & 2 & 0
    \end{bmatrix}^T
    \), \(
    \begin{bmatrix}
        1 & 0 & -1
    \end{bmatrix}^T
    \) and
    \( \begin{bmatrix}
        1 & 2 & 1
    \end{bmatrix}^T \).
\end{homeworkProblem}

\begin{homeworkProblem}
    Find the rank and the four eigenvalues of \( A \) and \( C \):

    \[
        A = \begin{bmatrix}
            1 & 1 & 1 & 1 \\
            1 & 1 & 1 & 1 \\
            1 & 1 & 1 & 1 \\
            1 & 1 & 1 & 1
        \end{bmatrix}
        \ \ 
        \text{and}
        \ \ 
        C = 
        \begin{bmatrix}
            1 & 0 & 1 & 0   \\
            0 & 1 & 0 & 1   \\
            1 & 0 & 1 & 0   \\
            0 & 1 & 0 & 1
        \end{bmatrix}.
    \]

    \solution
    \\
    
    \begin{enumerate}
        \item The rank of \( A \) is zero, the eigenvalues are 0, 0, 0, 4.
        \item The rank of \( C \) is 2, the eigenvalues are 0, 0, 2, 2.
    \end{enumerate}

\end{homeworkProblem}

\begin{homeworkProblem}
    Suppose \( A \) has eigenvalues 0,3,5 with independent eigenvectors \( \mathbf{u}, \mathbf{v}, \mathbf{w} \).
    \begin{enumerate}
        \item Give a basis for the nullspace and a basis for the column space.
        \item Find a particular solution to \( A\mathbf{x} = \mathbf{v} + \mathbf{w} \). Find all solutions.
        \item \( A\mathbf{x} = \mathbf{u} \) has no solution. If it did then \emptyunderline\ would be in the column space.
    \end{enumerate}

    \solution

    \begin{enumerate}
        \item A basis for the nullspace is \( \mathbf{u} \),a basis for the column space is \( \left\{ \mathbf{v}, \mathbf{w} \right\} \).
        \item A particular solution to \( A\mathbf{x} = \mathbf{v} + \mathbf{w} \) is \( \mathbf{x^*} = 1/3 \mathbf{v} + 1/5 \mathbf{w} \) and all solutions have the form \( \mathbf{x} = c\mathbf{u} + 1/3\mathbf{v} + 1/5\mathbf{w},\ c \in \mathbb{R} \).
        \item \( a\mathbf{v} + b\mathbf{w} \in\ C(A), a,b \in \mathbb{R} \).
    \end{enumerate}
\end{homeworkProblem}

\begin{homeworkProblem}
    \begin{enumerate}
        \item Factor these two matrices into \( A = X\Lambda X^{-1} \):
        \[
            A = \begin{bmatrix}
                1 & 2   \\
                0 & 3
            \end{bmatrix}
            \ \ \ \ \ \ 
            \text{and}
            \ \ \ \ \ \ 
            A = \begin{bmatrix}
                1 & 1   \\
                3 & 3   
            \end{bmatrix}
        \]
        \item If \( A = X\Lambda X^{-1} \) then \( A^3 = (\ )(\ )(\ ) \) and \( A^{-1} = (\ )(\ )(\ ) \).
    \end{enumerate}

    \solution

    \begin{enumerate}
        \item The factorization of first \( A = \begin{bmatrix}
            1 & 2   \\
            0 & 3
        \end{bmatrix} \) is \( A = \begin{bmatrix}
            1 & 1   \\
            0 & 1
        \end{bmatrix} 
        \begin{bmatrix}
            1 & 0   \\
            0 & 3
        \end{bmatrix}
        \begin{bmatrix}
            1 & -1   \\
            0 & 1
        \end{bmatrix}
        \), the factorization of second \( A = \begin{bmatrix}
            1 & 1   \\
            3 & 3
        \end{bmatrix} \)
        =
        \(
        \begin{bmatrix}
            -1 & 1   \\
            1 & 3
        \end{bmatrix}
        \begin{bmatrix}
            0 & 0   \\
            0 & 4
        \end{bmatrix}
        \begin{bmatrix}
            -3/4 & 1/4   \\
            1/4 & 1/4
        \end{bmatrix}
        \).
    \item \( A^3 = X\Lambda^3 X^{-1} \), \( A^{-1} = X\Lambda^{-1}X^{-1} \).
    \end{enumerate}

\end{homeworkProblem}

\begin{homeworkProblem}
    True or false: If the column of \( X \) (eigenvectors of \( A \)) are linearly independent, then
    
    \begin{enumerate}
        \item A is invertible.
        \item A is diagonalizable.
        \item X is invertible.
        \item X is diagonalizable.
    \end{enumerate}

    \solution

    \begin{enumerate}
        \item False. Consider \( \textbf{Problem 6} \).
        \item True.
        \item True.
        \item True.
    \end{enumerate}

\end{homeworkProblem}

\begin{homeworkProblem}
   \( A^k = X\Lambda^k X^{-1} \) approaches the zero matrix as \( k \to \infty \)
   if and only if every \( \lambda \) has absolute value less than \emptyunderline\ .
   Which of these matrices has \( A^k \to 0 \)?

   \[
        A_1 = \begin{bmatrix}
            .6 & .9 \\
            .4 & .1
        \end{bmatrix}
        \ \ \ 
        \text{and}
        \ \ \ 
        A_2 = \begin{bmatrix}
            .6 & .9 \\
            .1 & .6
        \end{bmatrix}
        .
   \]

   (Recommended) Find \( \Lambda \) and \( X \) to diagonalize \( A_1 \) in the above
   problem. What is the limit of \( \Lambda^k \) as \( k \to \infty \)? In the
   columns of this limiting matrix you see the \emptyunderline.
    \\

    \solution

    \begin{enumerate}
        \item \( |\lambda| < 1 \).
        \item The eigenvalues of \( A_1 \) are 1 and -0.3, the eigenvalues of \( A_2 \) are \( \frac{6 + \sqrt{11}}{10}, \frac{6-\sqrt{11}}{10} \), since
        the absolute value of the eigenvalues of \( A_2 \) are less than 1, we know \( A_2^k \to 0, k \to \infty \).
        \item \( A_1 \) can be diagonalized as \( A_1 = \begin{bmatrix}
            9 & -1  \\
            4 & 3
        \end{bmatrix}
        \begin{bmatrix}
            1 & 0   \\
            0 & -0.3
        \end{bmatrix} 
        \begin{bmatrix}
            1/13 & 1/13 \\
            -4/13 & 9/13
        \end{bmatrix} 
        \), \( \Lambda^k \to \begin{bmatrix}
            1 & 0   \\
            0 & 0
        \end{bmatrix} \)
        \\

        the columns with eigenvalues less than 1
        become the zero columns.
    \end{enumerate}
\end{homeworkProblem}

\begin{homeworkProblem}
    Show that trace \( XY = \text{trace} YX \) by adding the diagonal entries of \( XY \) and \( YX \):

    \[
        X = \begin{bmatrix}
            a & b \\
            c & d
        \end{bmatrix}
        \ \ \ \ \ \ 
        \text{and}
        \ \ \ \ \ \ 
        Y = \begin{bmatrix}
            q & r \\
            s & t
        \end{bmatrix}
    \]

    Now choose \( Y \) to be \( \Lambda X^{-1} \). Then \( X\Lambda X^{-1} \) has the same trace 
    as \( \Lambda X^{-1} X = \Lambda \). This proves that
    the trace of \( A \) equals the trace of \( \Lambda = \text{the sum of the eigenvalues} \).
    \( \mathbf{AB\ -\ BA = I}\ \textbf{is impossible} \) since the left side has trace \emptyunderline.
    \\

    \solution

    \begin{enumerate}
        \item \( {(XY)}_{11} + {(XY)}_{22} = (aq + bs) + (cr + dt) = (qa + rc) + (sb + td) = {(YX)}_{11} + {(YX)}_{22} \)
        \item 0.
    \end{enumerate}
\end{homeworkProblem}

\begin{homeworkProblem}
    \begin{enumerate}
        \item If \( A = \begin{bmatrix}
            a & b   \\
            0 & d
        \end{bmatrix} \) then the determinant of 
        \( A - \lambda I \) is \( (\lambda - a)(\lambda - d) \). Check the "Cayley-Hamilton Theorem"
        that \( (A - aI)(A - dI) = \text{zero matrix} \).
        \item Test the Cayley-Hamilton Theorem on Fibonacci's \( A = \begin{bmatrix}
            1 & 1   \\
            1 & 0
        \end{bmatrix} \). The theorem predicts
        that \( A^2 - A - I = 0 \), since the polynomial \( \det (A-\lambda I)\ \text{is}\ \lambda^2 - \lambda - 1 \).
        \\

        \solution

        \begin{enumerate}
            \item \( (A - aI)(A - dI) = \begin{bmatrix}
                0 & b   \\
                0 & d - a
            \end{bmatrix}
            \begin{bmatrix}
                a - d & b   \\
                0 & 0
            \end{bmatrix}
            = \mathbf{0}
            \).
            \item The characteristic polynomial of Fibonacci matrix is \( \lambda^2 - \lambda - 1 = 0 \), we have
            \[
                A^2 - A - I = \begin{bmatrix}
                    2 & 1   \\
                    1 & 1
                \end{bmatrix}
                -
                \begin{bmatrix}
                    1 & 1  \\
                    1 & 0
                \end{bmatrix}
                -
                \begin{bmatrix}
                    1 & 0   \\
                    0 & 1
                \end{bmatrix}
                =
                \begin{bmatrix}
                    0 & 0   \\
                    0 & 0
                \end{bmatrix}
            \]

            The result checks the Cayley-Hamilton Theorem is true.
        \end{enumerate}
    \end{enumerate}
\end{homeworkProblem}

\end{document}