\documentclass{article}

\usepackage{fancyhdr}
\usepackage{extramarks}
\usepackage{boondox-cal}
\usepackage{amsmath}
\usepackage{amsthm}
\usepackage{amsfonts}
\usepackage{tikz}
\usepackage[plain]{algorithm}
\usepackage{algpseudocode}

\usetikzlibrary{automata,positioning}

%
% Basic Document Settings
%

\topmargin=-0.45in
\evensidemargin=0in
\oddsidemargin=0in
\textwidth=6.5in
\textheight=9.0in
\headsep=0.25in

\linespread{1.1}

\pagestyle{fancy}
\lhead{\hmwkAuthorName}
\chead{\hmwkClass\ (\hmwkClassInstructor): \hmwkTitle}
\rhead{\firstxmark}
\lfoot{\lastxmark}
\cfoot{\thepage}

\renewcommand\headrulewidth{0.4pt}
\renewcommand\footrulewidth{0.4pt}

\setlength\parindent{0pt}

%
% Create Problem Sections
%

\newcommand{\enterProblemHeader}[1]{
    \nobreak\extramarks{}{Problem \arabic{#1} continued on next page\ldots}\nobreak{}
    \nobreak\extramarks{Problem \arabic{#1} (continued)}{Problem \arabic{#1} continued on next page\ldots}\nobreak{}
}

\newcommand{\exitProblemHeader}[1]{
    \nobreak\extramarks{Problem \arabic{#1} (continued)}{Problem \arabic{#1} continued on next page\ldots}\nobreak{}
    \stepcounter{#1}
    \nobreak\extramarks{Problem \arabic{#1}}{}\nobreak{}
}

\setcounter{secnumdepth}{0}
\newcounter{partCounter}
\newcounter{homeworkProblemCounter}
\setcounter{homeworkProblemCounter}{1}
\nobreak\extramarks{Problem \arabic{homeworkProblemCounter}}{}\nobreak{}

%
% Homework Problem Environment
%
% This environment takes an optional argument. When given, it will adjust the
% problem counter. This is useful for when the problems given for your
% assignment aren't sequential. See the last 3 problems of this template for an
% example.
%
\newenvironment{homeworkProblem}[1][-1]{
    \ifnum#1>0
        \setcounter{homeworkProblemCounter}{#1}
    \fi
    \section{Problem \arabic{homeworkProblemCounter}}
    \setcounter{partCounter}{1}
    \enterProblemHeader{homeworkProblemCounter}
}{
    \exitProblemHeader{homeworkProblemCounter}
}

%
% Homework Details
%   - Title
%   - Due date
%   - Class
%   - Section/Time
%   - Instructor
%   - Author
%

\newcommand{\hmwkTitle}{Homework\ \#2}
\newcommand{\hmwkDueDate}{February 24, 2024}
\newcommand{\hmwkClass}{Linear Algebra}
\newcommand{\hmwkClassInstructor}{Gilbert Strang}
\newcommand{\hmwkAuthorName}{\textbf{0130}}

%
% Title Page
%

\title{
    \vspace{2in}
    \textmd{\textbf{\hmwkClass:\ \hmwkTitle}}\\
    \vspace{0.1in}\large{\textit{\hmwkClassInstructor}}
    \vspace{3in}
}

\author{\hmwkAuthorName}
\date{}

\renewcommand{\part}[1]{\textbf{\large Part \Alph{partCounter}}\stepcounter{partCounter}\\}

%
% Various Helper Commands
%

% Useful for algorithms
\newcommand{\alg}[1]{\textsc{\bfseries \footnotesize #1}}

% For derivatives
\newcommand{\deriv}[1]{\frac{\mathrm{d}}{\mathrm{d}x} (#1)}

% For partial derivatives
\newcommand{\pderiv}[2]{\frac{\partial}{\partial #1} (#2)}

% Integral dx
\newcommand{\dx}{\mathrm{d}x}

% Alias for the Solution section header
\newcommand{\solution}{\textbf{\large Solution}}

% Probability commands: Expectation, Variance, Covariance, Bias
\newcommand{\E}{\mathrm{E}}
\newcommand{\Var}{\mathrm{Var}}
\newcommand{\Cov}{\mathrm{Cov}}
\newcommand{\Bias}{\mathrm{Bias}}

\begin{document}

\maketitle

\pagebreak

\begin{homeworkProblem}
    What multiple \( \mathcal{l} \) of equation 1 should be subtracted from equation 2 to
    remove \( c \)?
    
    \[
        \begin{split}
            ax + by = f
            \\
            cx + dy = g.
        \end{split}
    \]

    The first pivot is \( a \) (assumed nonzero). Elimination produces what formula
    for the second pivot? What is \( y \)? The second pivot is missing when \( ad = bc \): singular.
    \\

    \solution

    \begin{enumerate}
        \item \( \mathcal{l} = \frac{c}{a} \).
        \item formula: \( (d - \frac{c}{b}) y = g - \frac{c}{a}f \)
        \item \( y = \frac{ag-cf}{ad-bc}\)
    \end{enumerate}

\end{homeworkProblem}

\begin{homeworkProblem}
    For which three numbers \( k \) does elimination break down? Which is fixed
    by a row exchange? Is the number of solutions \( 0 \) or \( 1 \) or \( \infty \)?
    
    \[
        \begin{split}
            kx + 3y &= 6    \\
            3x + ky &= -6
        \end{split}
    \]

    \solution

    \begin{enumerate}
        \item \( k = 0, 3, -3 \)
        \item if \( k = 0 \), a row exchange can be used to fix the elimination.
        \item \begin{enumerate}
            \item \( k = 0 \) has \( 1 \) solution.
            \item \( k = 3 \) has \( 0 \) solution.
            \item \( k = -3 \) has \( \infty \) solutions.
        \end{enumerate}
    \end{enumerate}
\end{homeworkProblem}

\begin{homeworkProblem}
    Which number \( d \) forces a row exchange, and what is the
    triangular system (not singular) for that \( d \)? Which \( d \)
    makes this system singular (no third pivot) ?

    \[
        \begin{split}
            2x + 5y  + z &= 0   \\
            4x + dy + z &= 2    \\
            y - z &= 3
        \end{split}
    \]
    \\

    \solution

    \begin{enumerate}
        \item \( d = 10 \) forces a row exchange.
        \item \[
            \begin{split}
                \begin{bmatrix}
                    2 & 5 & 1   \\
                    0 & 1 & -1  \\
                    0 & 0 & -1
                \end{bmatrix}
            \end{split}
            \]
        \item \(d = 11\) makes the system singular.
    \end{enumerate}
\end{homeworkProblem}

\begin{homeworkProblem}
    Write down the 3 by 3 matrices that produce these elimination steps:
    
    \begin{enumerate}
        \item \( E_{21} \) subtracts 5 times row 1 from row 2.
        \item \( E_{32} \) subtracts -7 times row 2 from row 3.
        \item \( P \) exchanges rows 1 and 2, then rows 2 and 3.
    \end{enumerate}

    \solution
   
    \begin{enumerate}
        \item \[ E_{21} =
            \begin{bmatrix}
                1 & 0 & 0   \\
                -5 & 1 & 0  \\
                0 & 0 & 1
            \end{bmatrix}
            \]
        \item \[ E_{32} =
            \begin{bmatrix}
                1 & 0 & 0   \\
                0 & 1 & 0   \\
                0 & 7 & 1
            \end{bmatrix}
            \]
        \item \[ P =
            \begin{bmatrix}
                0 & 1 & 0   \\
                0 & 0 & 1   \\
                1 & 0 & 0
            \end{bmatrix}
            \]
    \end{enumerate}

\end{homeworkProblem}

\begin{homeworkProblem}
    Which three matrices \( E_{21}, E_{31}, E_{32} \) put \( A \)
    into triangular form \( U \)?

    \[
        A =
        \begin{bmatrix}
            1 & 1 & 0   \\
            4 & 6 & 1   \\
            -2 & 2 & 0
        \end{bmatrix}
    \]

    and \( E_{32}E_{31}E_{21}A = EA = U \).
    \\

    Multiply those \( E \)'s to get one elimination matrix \( E \).
    What is \( E^{-1} = L\)?
    \\

    Include \( \mathbf{b} = (1,0,0) \) as a fourth column to produce
    \( [A\ \mathbf{b}] \). Carry out the elimination steps on this augmented
    matrix to solve \( A\mathbf{x} = \mathbf{b} \).
    \\

    \solution

    \begin{enumerate}
        \item 
            \[ 
            E_{21} =
                \begin{bmatrix}
                    1 & 0 & 0   \\
                    -4 & 1 & 0  \\
                    0 & 0 & 1
                \end{bmatrix}
                ,
            E_{31} =
            \begin{bmatrix}
                1 & 0 & 0   \\
                0 & 1 & 0  \\
                2 & 0 & 1
            \end{bmatrix}
            ,
            E_{32} =
                \begin{bmatrix}
                    1 & 0 & 0   \\
                    0 & 1 & 0  \\
                    0 & -2 & 1
                \end{bmatrix}
            \]
        \item 
            \[
                E = E_{32}E_{31}E_{21} =
                \begin{bmatrix}
                    1 & 0 & 0   \\
                    -4 & 1 & 0  \\
                    10 & -2 & 1
                \end{bmatrix}
                ,
                L = E^{-1} =
                \begin{bmatrix}
                    1 & 0 & 0   \\
                    4 & 1 & 0  \\
                    0 & 2 & 1
                \end{bmatrix}
            \]
        \item 
            Apply the elimination steps to \( [ A\ \mathbf{b} ]\), then we get \( [U\ L\mathbf{b}] \):
            \[
                \begin{bmatrix}
                    1 & 1 & 0 & 1   \\
                    0 & 2 & 1 & 4   \\
                    0 & 0 & -2 & 0
                \end{bmatrix}
            \]
            
            then we know \( \mathbf{x} = {[-1, 2, 0]}^{T}\).
    \end{enumerate}

\end{homeworkProblem}

\begin{homeworkProblem}
    Suppose \( A \) is invertible and you exchange its first two rows to reach
    \( B \). Is the new matrix \( B \) invertible? How would you find \( B^{-1} \) from
    \( A^{-1} \)?
    \\

    \solution
    \\

    Yes, \( B \) is invertible. And since \( B = P_{12}A \), \( P_{12}^{-1} = P_{12}\),
    we have \( B^{-1} = A^{-1}P_{12} \), where \( P_{12} \) is the permutation matrix.
    And the formula implies that \( B^{-1} \) can be found by exchange the first two
    columns of \( A^{-1} \).
\end{homeworkProblem}

\begin{homeworkProblem}
    \begin{enumerate}
        \item What 3 by 3 matrix \( E \) has the same effect as these three steps? Subtract
                row 1 from row 2, subtract row 1 from row 3, then subtract row 2 from row 3.
        \item What single matrix \( L \) has the same effect as these three reverse steps? Add
                row 2 to row 3, add row 1 to row 3, then add row 1 to row 2.
    \end{enumerate}

    \solution
    
    \begin{enumerate}
        \item \[ E = 
            \begin{bmatrix}
                1 & 0 & 0   \\
                -1 & 1 & 0  \\
                0 & -1 & 1
            \end{bmatrix}
            \]
        \item \[ L = 
            \begin{bmatrix}
                1 & 0 & 0   \\
                1 & 1 & 0  \\
                1 & 1 & 1
            \end{bmatrix}
            \]
    \end{enumerate}
\end{homeworkProblem}

\begin{homeworkProblem}
    Prove that \( A \) is invertible if \( a \neq 0 \) and \( a \neq b \)
    (find the pivots of \( A^{-1} \)). Then find three numbers \( c \) so that
    \( C \) is not invertible.

    \[  A = 
        \begin{bmatrix}
            a & b & b   \\
            a & a & b   \\
            a & a & a
        \end{bmatrix}
        ,
        C = 
        \begin{bmatrix}
            2 & c & c   \\
            c & c & c   \\
            8 & 7 & c
        \end{bmatrix}
    \]
    \\

    \solution

    \begin{enumerate}
        \item \[ A^{-1} =
                \begin{bmatrix}
                   a & b & b   \\
                   0 & a-b & 0  \\
                   0 & 0 & a-b 
                \end{bmatrix}
            \]
        The pivots of \( A^{-1} \) are \( a, a-b, a-b \), so \( A \) is invertible.
        \item \( c = 0, 2, 7 \) makes \( C \) not invertible.
    \end{enumerate}
\end{homeworkProblem}

\begin{homeworkProblem}
    What three elimination matrices \( E_{21}, E_{31}, E_{32} \) put \( A \) into its
    upper triangular form \( E_{32}E_{31}E_{21}A = U\)? Multiply by \( E_{32}^{-1}, E_{31}^{-1}, E_{21}^{-1}\)
    to factor \( A \) into \( L \) times \( U \):

    \[
        A = 
        \begin{bmatrix}
            1 & 0 & 1   \\
            2 & 2 & 2   \\
            3 & 4 & 5
        \end{bmatrix}
    \]

and \( L = E_{21}^{-1}E_{31}^{-1}E_{32}^{-1} \).
\\

\solution

\[  E_{21} =
    \begin{bmatrix}
        1 & 0 & 0   \\
        -2 & 1 & 0  \\
        0 & 0 & 1
    \end{bmatrix}
    ,
    E_{31} =
    \begin{bmatrix}
        1 & 0 & 0   \\
        0 & 1 & 0  \\
        -3 & 0 & 1
    \end{bmatrix}
    ,
    E_{32} =
    \begin{bmatrix}
        1 & 0 & 0   \\
        0 & 1 & 0  \\
        0 & -2 & 1
    \end{bmatrix}
\]

then we can represent \( A \) as \( A = LU \), where

\[  L = 
    \begin{bmatrix}
        1 & 0 & 0   \\
        2 & 1 & 0  \\
        3 & 2 & 1
    \end{bmatrix}
    ,
    U = 
    \begin{bmatrix}
        1 & 0 & 1   \\
        0 & 2 & 0   \\
        0 & 0 & 2
    \end{bmatrix}
\]

\end{homeworkProblem}

\begin{homeworkProblem}
    \begin{enumerate}
        \item How many entries of \( S \) can be chosen
                independently, if \( S = S^T \) is 5 by 5?
        \item How do L and D (still 5 by 5) give the same
                number of entries in \( LDL^T \)?
        \item How many entries of \( A \) can be chosen 
                if \( A \) is skew-symmetric? (\(A^T = -A \)).
        \item Why does \( A^T A \) have no negetive numbers on its
                diagonal?
    \end{enumerate}

    \solution

    \begin{enumerate}
        \item 15.
        \item Ommited (Not sure what's the meaning of the question).
        \item Suppose \( A \) is \( n \) by \( n \), then the answer is \( n \).
        \item Because if we consider the i-i entry of \( A^T A \), we have:
                \[
                    {A^T A}_{ii} = \sum_{j=1}^{n} {A^T}_{ij} A_{ji} = \sum_{j=1}^{n} {A_{ji}}^2 \geq 0
                \]
    \end{enumerate}
\end{homeworkProblem}

\end{document}
